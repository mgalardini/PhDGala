%%%%%%%%%%%%%%%%%%%%%%%%%%%%%%%%%%%%%%%%%%%%%%
\logvartrue
\chapter{Conclusions}
%%%%%%%%%%%%%%%%%%%%%%%%%%%%%%%%%%%%%%%%%%%%%%

The awareness about the importance of the microbial world for the higher organisms and ultimately for the human race has grown in the last decade, together with the emergence of the genomics and comparative genomics sciences. The challenges posed by a growing world population and the need for a more efficient agriculture with less waste of energy and land need to be addressed using also the vast resources that can be extracted from natural bacterial populations; the nitrogen fixation process of the \textit{Sinorhizobium} - \textit{Medicago} symbiosis it's an excellent example of a natural process that would allow a more productive and efficient agriculture, process which can be optimized through the application of a comparative genomics approach (such as gene mining).

In the first part of this thesis the development of two computational tools needed to help the comparative genomics analysis is described: the need to extract the maximum amount of information from complex draft genomes is matched by CONTIGuator, while the need to give genetic explanations to phenotypes in an automatic fashion is matched by DuctApe. Both softwares have been developed to be applied to similar problems in different species than \textit{S. meliloti} and to be widely adapted by the scientific community through a focus on usability and a clear set of results.

In the second part of the thesis the presence of a common genetic repertoire for the association of \textit{Alphaproteobacteria} with plants has been tested through a comparative genomics on a large and heterogenuos set of genomes: indeed species that are associated with plants tend to have a larger genome, putatively harboring genetic elements needed for the establishment of a successful association, either through endophytism or symbiosis: the broad range of association mechanisms lead to the impossibility to find a large repertoire of shared genetic elements for endophytism, while the genes that are most likely needed for the symbiotic process are indeed more and more defined, including for instance nodulating genes; from this analysis we can conclude that the plant symbiosis phenotype has a conserved gene repertoire inside \textit{Alphaproteobacteria}.

In the last part of the thesis, two comparative genomics studies on \textit{Sinorhizobium meliloti} have been carried on to understand if the variability in the plant growth promotion phenotype was mirrored by a variability in the genetic elements related to the symbiotic process and to understand the evolution and functional features of the peculiar genomic structure in this species. Indeed, both the presence/absence pattern of symbiotic genes and regulatory elements could be related to the observed phenotypic differences, leading to the definition of a large accessory gene set that could be furtherly tested in genetic engineering experiments towards the creation of new \textit{S. meliloti} strains. The three replicons that compose the genome of this species have also been characterized from a functional and evolutionary point of view, showing that the pSymB chromid has an important role in intra-specific differentiation and functional evolution through positive selection, while the pSymA megaplasmid has an important role in structural evolution and in the emergence of new functions.

Even though the genomics era has posed many new challenges, especially to computational biologists, the creation and application of the new analysis methods of the comparative genomics has proven to be able to help in the definition of the most probable elements that can explain functional and evolutionary differences found in the vast reservoir of diversity that is the bacterial community, a rich and promising field that needs to be studied for a better understanding of the bacterial world and its economical, agricultural and environmental impact on human life.